% Created 2010-12-17 Sex 00:03
\documentclass[11pt]{article}
\usepackage[utf8]{inputenc}
\usepackage[T1]{fontenc}
\usepackage{fixltx2e}
\usepackage{graphicx}
\usepackage{longtable}
\usepackage{float}
\usepackage{wrapfig}
\usepackage{soul}
\usepackage{t1enc}
\usepackage{textcomp}
\usepackage{marvosym}
\usepackage{wasysym}
\usepackage{latexsym}
\usepackage{amssymb}
\usepackage{hyperref}
\tolerance=1000
\providecommand{\alert}[1]{\textbf{#1}}

\title{Análise de Compositores por Período}
\author{}
\date{17 dezembro 2010}

\begin{document}

\maketitle


\section{Características}
\label{sec-1}
\subsection{Sacro x Profano}
\label{sec-1_1}

Considera-se música sacra aquela ligada a ofícios religiosos, enquanto que a produção profana é aquela em que não há essa ligação. 
\subsection{Duração curta x Longa}
\label{sec-1_2}

Produção com ênfase em obras curtas ou longas (maiores que 20 minutos).
\subsection{Harmonia x Contraponto}
\label{sec-1_3}

Discurso musical baseado (com ênfase) no discurso harmônico ou
no contraponto. A ênfase no discurso harmônico aqui
considerado prioriza uma só melodia.
\subsection{Vocal x Instrumental}
\label{sec-1_4}

Produção prioritariamente vocal ou instrumental.
\subsection{Razão x Emoção}
\label{sec-1_5}

Preocupação com o conjunto das características da música ou apelo emocional.
\subsection{Constância motívica x Variedade}
\label{sec-1_6}

Equilíbrio encontrado para repetição/reutilização de motivos melódicos e para
a utilização de novos materiais.
\subsection{Ritmo x Poliritmo}
\label{sec-1_7}

Presença de poliritmia (quiálteras de subdivisões diferentes).
\subsection{Const. Tonal x Modulação}
\label{sec-1_8}

Ritmo de variação de tonalidade em uma secção de peça.
\section{Notas}
\label{sec-2}



\begin{center}
\begin{tabular}{lrrrrlrrr}
 Compositor      &  S-P  &           C-L  &  H-C  &           V-I  &  N-D           &  M-V  &           R-P  &  T-M  \\
\hline
 1) Bach         &  2.0  &           6.0  &  9.0  &           2.0  &  \textbf{8.0}  &  2.0  &           1.0  &  5.0  \\
 2) Mozart       &  6.0  &           4.0  &  1.0  &           6.0  &  \textbf{6.0}  &  7.0  &           2.0  &  2.0  \\
 3) Beethoven    &  7.0  &           8.0  &  2.5  &           8.0  &  \textbf{5.0}  &  4.0  &           4.0  &  7.0  \\
 4) Chopin       &  9.0  &  \textbf{3.0}  &  3.0  &           9.0  &  \textbf{5.5}  &  8.0  &           7.0  &  8.0  \\
 5) Brahms       &  6.0  &           6.0  &  4.0  &  \textbf{7.0}  &  \textbf{4.5}  &  6.5  &  \textbf{5.0}  &  7.0  \\
 6) Stravinsky   &  8.0  &           7.0  &  6.0  &  \textbf{7.0}  &  \textbf{8.0}  &  5.0  &           8.0  &  5.0  \\
 7) Stockhausen  &  7.0  &           4.0  &  8.0  &           7.0  &  \textbf{5.0}  &  8.0  &           9.0  &  6.0  \\
\end{tabular}
\end{center}
\section{Matriz de Correlação}
\label{sec-3}



\begin{center}
\begin{tabular}{lrrrrrrrr}
      &    S-P  &    C-L  &    H-C  &    V-I  &    N-D  &    M-V  &    R-P  &    T-M  \\
 S-P  &    1.0  &  -0.22  &  -0.48  &   0.95  &  -0.39  &   0.69  &   0.76  &   0.37  \\
 C-L  &  -0.22  &    1.0  &    0.1  &  -0.15  &   0.22  &  -0.74  &   -0.2  &   0.09  \\
 H-C  &  -0.48  &    0.1  &    1.0  &  -0.58  &   0.47  &  -0.37  &   0.18  &   0.09  \\
 V-I  &   0.95  &  -0.15  &  -0.58  &    1.0  &  -0.62  &   0.68  &   0.65  &    0.5  \\
 N-D  &  -0.39  &   0.22  &   0.47  &  -0.62  &    1.0  &  -0.61  &  -0.23  &  -0.44  \\
 M-V  &   0.69  &  -0.74  &  -0.37  &   0.68  &  -0.61  &    1.0  &    0.6  &   0.12  \\
 R-P  &   0.76  &   -0.2  &   0.18  &   0.65  &  -0.23  &    0.6  &    1.0  &   0.45  \\
 T-M  &   0.37  &   0.09  &   0.09  &    0.5  &  -0.44  &   0.12  &   0.45  &    1.0  \\
\end{tabular}
\end{center}
\section{Autovalores}
\label{sec-4}



\begin{center}
\begin{tabular}{rr}
 Autovalor  &        Valor  \\
\hline
         1  &  51.13889085  \\
         2  &    18.399836  \\
         3  &  15.89492118  \\
         4  &  10.24835136  \\
         5  &   4.08532881  \\
         6  &   0.23267181  \\
         7  &           0.  \\
\end{tabular}
\end{center}
\section{Contribuições de cada característica}
\label{sec-5}



\begin{center}
\begin{tabular}{lrr}
 Características  &               C$_1$  &               C$_2$  \\
\hline
 S-P              &  16.525754968461332  &  4.0766395424150526  \\
 C-L              &  7.4914596976226395  &  18.707205119314175  \\
 H-C              &  8.8296961304835495  &  18.403433605321421  \\
 V-I              &  17.272916504205682  &  3.7951383008863879  \\
 N-D              &   12.68048219897665  &  4.2624011225807896  \\
 M-V              &  15.658827206283824  &  11.667117892269376  \\
 R-P              &  12.916471308915309  &  16.157726001664628  \\
 T-M              &  8.6243919850510053  &   22.93033841554816  \\
\end{tabular}
\end{center}
\section{Oposições (inovação)}
\label{sec-6}



\begin{center}
\begin{tabular}{lrr}
 Movimento Musical             &    W$_{\mathrm{i,j}}$  &      S$_{\mathrm{i,j}}$  \\
\hline
 Bach $\to$ Mozart             &   0.99999999999999989  &  5.2369115333442702e-16  \\
 Mozart $\to$ Beethoven        &   0.59007230475630668  &      2.1701394155920082  \\
 Beethoven $\to$ Chopin        &  -0.11952016955511029  &      2.5737985445176159  \\
 Chopin $\to$ Brahms           &   0.33856095460092911  &     0.43697862189129016  \\
 Brahms $\to$ Stravinsky       &   0.34204825316049009  &      1.2119607469777283  \\
 Stravinsky $\to$ Stockhausen  &   0.44876907953599465  &      1.4120279905772009  \\
\end{tabular}
\end{center}
\section{Dialética}
\label{sec-7}



\begin{center}
\begin{tabular}{lr}
 Tripla Musical                             &  D$_{\mathrm{i \to k}}$  \\
\hline
 Bach $\to$ Mozart $\to$ Beethoven          &     0.15241265495976331  \\
 Mozart $\to$ Beethoven $\to$ Chopin        &     0.35743935239030455  \\
 Beethoven $\to$ Chopin $\to$ Brahms        &     0.25593899443066082  \\
 Chopin $\to$ Brahms $\to$ Stravinsky       &      1.0384303391187353  \\
 Brahms $\to$ Stravinsky $\to$ Stockhausen  &     0.70485214163314314  \\
\end{tabular}
\end{center}

\end{document}