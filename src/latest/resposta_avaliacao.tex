% Created 2011-05-31 Ter 16:08
\documentclass[11pt]{article}
\usepackage[utf8]{inputenc}
\usepackage[T1]{fontenc}
\usepackage{fixltx2e}
\usepackage{graphicx}
\usepackage{longtable}
\usepackage{float}
\usepackage{wrapfig}
\usepackage{soul}
\usepackage{textcomp}
\usepackage{marvosym}
\usepackage{wasysym}
\usepackage{latexsym}
\usepackage{amssymb}
\usepackage{hyperref}
\tolerance=1000
\providecommand{\alert}[1]{\textbf{#1}}

\title{Reposta a Avaliação}
\author{}
\date{\today}

\begin{document}

\maketitle


\section{Análise de Agentes Aleatórios}
\label{sec-1}


  Autovalores para análise dos filósofos:


\begin{center}
\begin{tabular}{lrrrrrrr}
 Filósofos  &  51.25  &  23.56  &  16.69  &  4.76  &  2.64  &  1.07  &  0.  \\
\end{tabular}
\end{center}



  Foram replicados cada teste 5 vezes.

  Para número de agentes $n_a = 7$ e número de características $n_c = 8$:


\begin{center}
\begin{tabular}{lrrrrrrr}
 Teste 1  &  37.07  &  24.69  &  19.34  &  13.10  &  4.04  &  1.73  &  0.  \\
 Teste 2  &  49.28  &  26.71  &  11.24  &   8.78  &  3.56  &  0.40  &  0.  \\
 Teste 3  &  47.51  &  23.48  &  14.58  &  10.51  &  3.13  &  0.75  &  0.  \\
 Teste 4  &  42.31  &  32.58  &  11.98  &   6.49  &  4.59  &  2.03  &  0.  \\
 Teste 5  &  37.43  &  28.68  &  17.62  &  10.46  &  5.56  &  0.22  &  0.  \\
\end{tabular}
\end{center}



  Para número de agentes $n_a = 20$ e número de características $n_c = 8$:


\begin{center}
\begin{tabular}{lrrrrrrrr}
 Teste 1  &  24.80  &  18.90  &  17.13  &  15.01  &   9.85  &  6.58  &  5.21  &  2.48  \\
 Teste 2  &  23.82  &  20.65  &  18.67  &  13.09  &   8.36  &  7.65  &  4.89  &  2.83  \\
 Teste 3  &  22.72  &  19.84  &  15.66  &  12.52  &  11.67  &  7.56  &  5.61  &  4.38  \\
 Teste 4  &  23.71  &  19.88  &  17.74  &  13.49  &   8.72  &  7.23  &  6.56  &  2.64  \\
 Teste 5  &  23.11  &  20.64  &  17.00  &  13.16  &   9.44  &  8.49  &  5.02  &  3.10  \\
\end{tabular}
\end{center}



  Para número de agentes $n_a = 100$ e número de características $n_c = 8$:


\begin{center}
\begin{tabular}{lrrrrrrrr}
 Teste 1  &  17.01  &  14.74  &  14.33  &  13.28  &  12.33  &  11.70  &  9.47  &  7.09  \\
 Teste 2  &  17.83  &  16.99  &  14.88  &  12.74  &  11.47  &  10.77  &  8.46  &  6.82  \\
 Teste 3  &  19.73  &  16.62  &  14.27  &  12.48  &  11.29  &  10.38  &  8.61  &  6.59  \\
 Teste 4  &  17.79  &  16.01  &  14.71  &  13.22  &  12.00  &  11.06  &  8.27  &  6.90  \\
 Teste 5  &  17.34  &  16.09  &  13.99  &  12.57  &  12.06  &  10.16  &  9.47  &  8.29  \\
\end{tabular}
\end{center}



  Conclusão: Há concentração próxima a $70 \%$ quando $n_a = 7$. Conforme aumentamos o número de agentes,
  a concentração se espalha mais uniformemente.
\section{Normalização das notas}
\label{sec-2}


  Utilizando $Z = (X - mean)/std)$.

  Notas normalizadas:


\begin{center}
\begin{tabular}{lrrrrrrrr}
 Plato      &  -0.78  &  -0.59  &   1.53  &  -0.01  &  -0.20  &  -0.59  &  -0.01  &  -0.20  \\
 Aristotle  &   1.14  &   0.95  &   0.76  &   0.18  &   0.95  &   1.14  &  -0.97  &  -0.97  \\
 Descartes  &  -1.36  &  -0.97  &   1.53  &   0.56  &   0.76  &  -0.97  &   0.95  &   0.95  \\
 Espinoza   &   1.14  &  -1.17  &  -1.55  &  -0.01  &  -1.17  &  -0.78  &  -1.55  &  -1.55  \\
 Kant       &   0.76  &  -0.97  &   1.34  &   0.56  &  -0.20  &  -0.59  &   0.95  &  -0.01  \\
 Nietzsche  &   0.95  &   1.53  &  -1.55  &   1.53  &  -0.01  &   1.14  &  -1.55  &  -1.36  \\
 Deleuze    &   0.18  &   0.95  &  -1.55  &   1.14  &  -0.97  &   0.18  &  -0.01  &   0.37  \\
\end{tabular}
\end{center}



  Diferenças entre pares consecutivos:


\begin{center}
\begin{tabular}{lrr}
 Par de filósofos        &  Distância não normalizada  &  Distância normalizada  \\
 Plato -> Aristotle      &                       23.5  &                   9.08  \\
 Aristotle -> Descartes  &                       30.5  &                  11.79  \\
 Descartes -> Espinoza   &                       35.0  &                  13.53  \\
 Espinoza -> Kant        &                       24.0  &                   9.27  \\
 Kant -> Nietzsche       &                       32.0  &                  12.37  \\
 Nietzsche -> Deleuze    &                       18.0  &                   6.95  \\
\end{tabular}
\end{center}



  Diferenças entre pares questionados pelo avaliador:


\begin{center}
\begin{tabular}{lrr}
 Par de filósofos      &  Distância não normalizada  &  Distância normalizada  \\
 Plato -> Kant         &                       10.0  &                   3.86  \\
 Aristotle -> Deleuze  &                       24.5  &                   9.47  \\
\end{tabular}
\end{center}



  Conclusão: as diferenças entre pares relevantes (os pares consecutivos) permanecem
  quando avaliando os resultados após aplicação do PCA.

\end{document}